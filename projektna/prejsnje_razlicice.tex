\paragraph{}Da bi aplikacija delovala kar najbolje in uporabnikom ponujala čim več podatkov, je njen razvoj do danes obsegal tri večje različice, vsaka med njimi pa tudi vrsto manjših sprotnih popravkov. Vsaka različica je korenito spremenila način pridobivanja, obdelovanja in prikazovanja podatkov.

\subsection{Različica 1.0}
\paragraph{}Prva različica aplikacije je prikazovala izključno nadomeščanja v preprosti besedilni obliki, ki jih je pridobila iz elektronske redovalnice Gimnazije Vič. Omogočala je osnovne filtre za posamezne razrede in omejen nabor filtrov za profesorje. Njeni glavni pomanjkljivosti sta bili:
\begin{itemize}
  \setlength\itemsep{0em}
  \item obdelava velike količine podatkov kar na mobilni napravi, zaradi česar je aplikacija delovala počasi in se včasih celo sesula,
  \item filtriranje z enim samim pogojem (razredom), kar pomeni, da uporabnik sploh ni mogel videti nadomeščanj za svoje izbirne predmete (to je veljalo predvsem za dijake višjih letnikov).
\end{itemize}

\subsection{Različica 2.0}
\paragraph{} Aplikacija GimVic 2.0 je sledila kakšno leto po svoji predhodnici. Da bi podatke o nadomeščanjih bolje umestili v kontekst, je ta verzija prikazovala celoten urnik z nadomeščanji. Izboljšan je bil tudi sistem filtriranja podatkov, ki je uporabniku omogočal prikazovanje njegovih izbirnih predmetov. Aplikacija pa je podatke še vedno obdelovala na telefonu. Teh je bilo zdaj še več, tako da so se na starejših napravah pojavljale težave s pomanjkanjem pomnilnika.

\subsection{Različica 3.0}
\paragraph{} Zadnja različica aplikacije je bila zasnovana z mislijo na dotedanje pomankljivosti. Podatki se obdelujejo na ločenem strežniku za vse uporabnike naenkrat, da se izognemo potratni in nepotrebni obdelavi podatkov na vsakem telefonu posebej. Poleg tega omogoča tudi prikazovanje jedilnika.

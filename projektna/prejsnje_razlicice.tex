\paragraph{}Da bi aplikacija delovala kar najbolje in uporabnikom nudila čim več podatkov, je njen razvoj do danes obsegal 3 večje različice in vrsto manjših popravkov.

\subsection{Različica 1.0}
\paragraph{}Prva različica aplikacije je prikazovala izključno nadomeščanja v tekstovni obliki, ki jih je pridobila iz Spletne redovalnice Gimnazije Vič. Omogočala je osnovne filtre za posamezne razrede in omejen nabor filtrov za profesorje. Njeni glavni pomankljivosti sta bili:
\begin{itemize}
  \setlength\itemsep{0em}
  \item obdelava velike količine podatkov kar na maobilni napravi, kar je vodilo v počasnost aplikacije in včasih celo sesutje,
  \item filtriranje po enem samem razredu naenkrat, kar pomeni, da uporabnik ne more videti nadomeščanj za svoje izbirne predmete.
\end{itemize}

\subsection{Različica 2.0}
\paragraph{} Aplikacija GimVic 2.0 je sledila kakšno leto po svoji predhodnici. Da bi podatke o nadomeščanjih bolje umestili v kontekst, je ta verzija prikazovala tudi urnik. Izboljšal sem tudi sistem filtriranja podatkov. Aplikacija pa je podatke še vedno obdelovala na telefonu. Teh je bilo zdaj še več, tako da so se na počasnejših napravah pojavljale velike težave.

\subsection{Različica 3.0}
\paragraph{} Zadnja različica aplikacije s svojim delovanjem reši vse dotedanje težave, poleg tega pa prikazuje še jedilnik. Podatki se obdelujejo na ločenem strežniku za vse uporabnike naenkrat, da se izognemo potratni in nepotrebni obdelavi podatkov na vsakem telefonu posebej.

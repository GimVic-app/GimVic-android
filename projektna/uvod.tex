\paragraph{}O morebitnih spremembah v organizaciji pouka in šolske prehrane je med šolskim letom treba sproti obveščati profesorje in dijake. Na Gimnaziji Vič so informacije o tem na voljo v spletni učilnici ter na oglasni deski v šoli. Ker pa sta oba načina pregledovanja nadomeščanj nekoliko zamudna, sem se že v prvem letniku odločil napisati aplikacijo za lažje pregledovanje teh podatkov kar na pametnem telefonu, ki jih večina dijakov in profesorjev v vsakdanjem življenju pogosto uporablja. Ker je med njimi največ takih z operacijskim sistemom Android\cite{android-wiki}, je aplikacija napisana za to platformo.  Zaradi pomankanja izkušenj se takrat problema še nisem lotil celovito. Aplikacijo sem zato kasneje še večkrat popravil in celo dvakrat napisal na novo. Razvoj zadnje različice 3.0 je opisan v tej seminarski nalogi.

\paragraph{} Preden sem se lotil razvoja aplikacije, sem si zadal določene cilje. Aplikacija GimVic 3.0 je morala dijakom in profesorjem Gimnazije Vič prikazovati urnik, nadomeščanja in jedilnik. Ker je javno objavljena, mora biti dostop do urnika zaradi varovanja zasebnosti po dogovoru z ravnateljico mag. Alenko Krapež za dijake omejen z največ petkratno izbiro razreda, za profesorje pa zaščiten z geslom. Ker aplikacija za delovanje potrebuje ažurne podatke, je za osvežitev podatkov nujna internetna povezava.

\paragraph{} Vsi zastavljeni cilji so bili tudi doseženi. Svojo uspešnost sem na koncu preveril še z obdelavo statističnih podatkov, pridobljenih z Googlove spletne strani za razvijalce.

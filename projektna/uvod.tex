\paragraph{} Skozi celotno šolsko leto se na Gimnaziji Vič za dijake in profesorje pojavljajo nadomeščanja in druge spremembe v urniku. Ta so za dijake na voljo na spletni učilnici Gimnazije Vič ter na oglasni deski v šoli. Ker pa sta oba načina pregledovanja nadomeščanj nekoliko zamudna, sem se že v prvem letniku odločil napisati Android\cite{android-wiki} aplikacijo za lažje pregledovanje teh podatkov. Moj pristop k problemu je bil zaradi pomankanja izkušenj takrat še zelo zaletav. Aplikacijo sem zato kasneje še večkrat popravil in celo dvakrat napisal od začetka. Razvoj zadnje različice 3.0 je opisan v tej seminarski nalogi.

\paragraph{} Preden sem se lotil razvoja aplikacije, sem si zadal določene cilje. Aplikacija GimVic 3.0 je morala dijakom in profesorjem Gimnazije Vič prikazovati urnik, nadomeščanja in jedilnik. Dostop do urnika mora biti po dogovoru z ravnateljico mag. Alenka Krapež za dijake omejen z največ petkratnim spreminjanjem razreda, za profesorje pa zaščiten z geslom.

\paragraph{} Vsi zastavljeni cilji so bili tudi doseženi. Svojo uspešnost sem na koncu preveril še z obdelavo statističnih podatkov, pridobljenih iz Googlove spletne strani za razvijalce.

\paragraph{}Android je Googlov odprtokodni\cite{android-source} operacijski sistem mobilne naprave. Zgrajen je na Linuksovem jedru\cite{linux-kernel-wiki} in je bil prvotno namenjen uporabi na mobilnih telefonih. Kasneje se začel uporabljati tudi na tabličnih računalnikih, v zadnjem času pa celo na prenosnih računalnikih, urah in televizijah.

\subsection{Programiranje za platformo Android}
\paragraph{}Za razvoj aplikacij za platformo Android lahko uporabimo najrazličnejša razvijalska orodja, najpogosteje uporabljeno pa je Android Studio. Tega je razvilo podjetje Google in je osnovano na univerzalnem razvojnem orodju Intellij Idea\cite{intellij-idea} podjetja Jetbrains. Poleg platforme za razvoj aplikacij pa Google ponuja tudi spletno trgovino Google Play, kamor lahko razvijalci tudi objavimo svoje aplikacije.

\subsubsection{Programski jeziki}
\paragraph{}Za platformo Android lahko programiramo v več različnih jezikih, med njimi Java, Bash, C, C++ ter nekateri spletni jeziki (HTML, JavaScript, CSS), v zadnjem času pa celo Go in Python. Daleč najpogosteje se uporablja Java, saj je zanjo pripravljen zelo obširen nabor knjižnjic za interakcijo z uporabnikom ter cel kup orodji za prevajanje in sestavljanje posameznih delov v zaokroženo celoto imenovano aplikacija.

\subsubsection{Struktura aplikacije}
\paragraph{}Izvorna koda Android aplikacije je sestavljena iz več datotek in map. Te pa se delijo na 3 pomembnejše tipe:
\begin{itemize}
  \setlength\itemsep{0em}
  \item {\bf Manifest} datoteke so zapisane v formatu xml in operacijskemu sistemu povedo, kaj aplikacija počne, potrebuje, in zagotavljajo ostale podatke o njeni strukturi.
  \item {\bf Java} datoteke so izvorna koda v programskem jeziku Java, ki se izvaja na napravi.
  \item {\bf Res} ali {\bf resource} so datoteke, ki jih aplikacija potrebuje za prikazovanje uporabniškega vmesnika. To vključuje tekstovne datoteke xml, ki vsebujejo najrazličnejša besedila in opis izgleda aplikacije ter datoteke, ki vsebujejo slike, video posnetke in podobno gradivo.
\end{itemize}

\paragraph{}Med izvajanjem se aplikacija deli na posamezne niti. Vsaka nit opravlja svojo nalogo, ena iz med njih pa je glavna nit. Ta je prva, ki jo operacijski sistem požene in skrbi za zaganjanje ter ustavljanje vseh ostalih. Njena primarna naloga je skrb za uporabniški vmesnik. Zato ta nit ne sme opravljati nobenih dajših in zahtevnejših opravil, sicer aplikacija zastane, opracijski sistem pa jo zaradi tega po določenem času ubije.

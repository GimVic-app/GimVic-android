\paragraph{}Da bi preveril svojo, uspešnost sem za konec še analiziral statistične podatke o uporabi aplikacije. Te lahko pridobimo iz razvijalske konzole trgovine Google Play. To vključuje podatke o številu naprav, na katerih je nameščena aplikacija, številu aktivnih uporabnikov ter podatke o morebitnih zrušitvah aplikacije.

\paragraph{}Trenutno je aplikacija nameščena na 746 napravah, od tega je aktivnih uporabnikov 559. Operacijski sistem Android po vsakem sesutju katerekoli aplikacije trgovini Google Play pošlje poročilo, ki vsebuje razlog sesutja in tehnične specifikacije naprave. V zadnjem mesecu ni bilo prejetno nobeno poročilo o sesutju aplikacije GimVic, zato lahko upravičeno sklepamo, da deluje stabilno in po pričakovanjih.

\paragraph{}Število aktivnih uporabnikov presega vsa moja pričakovanja. Na Gimnaziji Vič je pribljižno 800 dijakov, ki pa imajo najrazličnejše mobilne telefone. Po mojih ocenah imajo torej praktično vsi dijaki, ki imajo telefon Android, nameščeno tudi aplikacijo GimVic, iz česar sklepam, da so z njenim delovanjem zadovoljni.
